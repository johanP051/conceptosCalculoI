\section{Geometría Cartesiana}
En las secciones 1.1, 1.2 y 1.3 del libro \cite{4} se encuentran los conceptos básicos sobre plano Cartesiano y funciones:

\section*{Ideas básicas}
\begin{itemize}
    \item La geometría cartesiana es fundamental para el desarrollo del cálculo integral, ya que permite representar puntos en el plano mediante coordenadas numéricas.
    \item EL plano Cartesiano es un sistema de referencia, es decir, permite ubicar puntos mediante coordenadas, fue introducido por René Descartes (1596-1650) y se basa en el uso de un sistema de ejes perpendiculares.
\end{itemize}

Para el sistema de coordenadas cartesianas:
\begin{itemize}
    \item Se utilizan dos rectas perpendiculares: una horizontal (\textbf{eje x}) y otra vertical (\textbf{eje y}).
    \item El punto de intersección de los ejes es el \textbf{origen (O)}.
    \item Cada punto en el plano se representa mediante un par ordenado $(x,y)$, donde:
    \begin{itemize}
        \item $x$ es la \textbf{abscisa} (distancia al eje y).
        \item $y$ es la \textbf{ordenada} (distancia al eje x).
    \end{itemize}
    \item Los signos de las coordenadas determinan el cuadrante en el que se encuentra el punto:
    \begin{itemize}
        \item \textbf{Primer cuadrante}: $x > 0, y > 0$
        \item \textbf{Segundo cuadrante}: $x < 0, y > 0$
        \item \textbf{Tercer cuadrante}: $x < 0, y < 0$
        \item \textbf{Cuarto cuadrante}: $x > 0, y < 0$
    \end{itemize}
\end{itemize}

\subsubsection*{Espacio tridimensional}
\begin{itemize}
    \item En el espacio, se utilizan tres ejes perpendiculares $(x, y, z)$ que se cortan en el origen.
    \item Cada punto en el espacio se describe mediante tres coordenadas $(x, y, z)$.
\end{itemize}

\subsubsection*{Aplicaciones geométricas}
\begin{itemize}
    \item La ecuación de la circunferencia $r^2 = x^2 + y^2$, representa un punto $(x, y)$ que está a una distancia $|x|$ unidades desde el origen haste el eje vertical y una distancia $|y|$ desde el origen hasta el eje horizontal.
\end{itemize}

\subsection*{Funciones. Ideas generales y ejemplos}

De manera informal se puede decir que asigna a cada elemento $x$ de un conjunto $X$ (dominio) un único elemento $y$ de otro conjunto $Y$ (imagen).

Las funciones pueden representarse de varias maneras:
\begin{itemize}
    \item \textbf{Gráficas}: Los puntos $(x, f(x))$ forman la gráfica de la función en el plano cartesiano.
    \item \textbf{Tablas}: Muestran pares $(x, y)$ de manera explícita.
    \item \textbf{Fórmulas}: Expresan la relación entre $x$ y $y$ mediante ecuaciones algebráicas.
\end{itemize}

\subsubsection*{Ejemplos de funciones}
\begin{itemize}
    \item \textbf{Función identidad}: $f(x) = x$. Su gráfica es una recta que pasa por el origen y forma ángulos iguales con los ejes.
    \item \textbf{Función número primo}: $\pi(x)$, que cuenta los números primos menores o iguales a $x$. Su gráfica consiste en segmentos horizontales con saltos en los números primos.
    \item \textbf{Función factorial}: $f(n) = n!$, que calcula el producto de todos los enteros positivos hasta $n$.
    \item \textbf{Función valor absoluto}: $\phi(x) = |x|$. Asigna a cada número real $x$ su valor absoluto.
\end{itemize}

\subsubsection*{Propiedades del valor absoluto}

\begin{multicols}{2}
\begin{enumerate}[label=(\arabic*)]
    \item
    \[
        |x| = \sqrt{x^2}
    \]

    \item
    \[
        |x|^2 = x^2
    \]

    \item
    \[
        x + b \leq |x + b|
    \]

    \item
    \[
        |x - b| = |b - x|
    \]

    \item
    \[
        |x| \geq 0
    \]

    \item
    \[
        |xy| = |x|\cdot|y|
    \]

    \item
    \[
        \left|\frac{x}{y}\right| = \frac{|x|}{|y|} 
        \quad (\text{con } y \neq 0)
    \]

    \item
    \[
        |x + y| \leq |x| + |y|
    \]

    \item
    \[
        ||x| - |y|| \leq |x - y|
    \]

    \item
    \[
        |x| < a 
        \;\;\Leftrightarrow\;\; -a < x < a 
        \quad (\text{para } a > 0)
    \]

    \item
    \[
        |x| \leq a 
        \;\;\Leftrightarrow\;\; -a \leq x \leq a 
        \quad (\text{para } a > 0)
    \]

    \item
    \[
        |x| > a 
        \;\;\Leftrightarrow\;\; x < -a \;\text{o}\; x > a 
        \quad (\text{para } a > 0)
    \]

    \item
    \[
        |x| \geq a 
        \;\;\Leftrightarrow\;\; x \leq -a \;\text{o}\; x \geq a 
        \quad (\text{para } a > 0)
    \]
\end{enumerate}
\end{multicols}

\subsubsection*{Propiedades de las funciones}
\begin{itemize}
    \item Para cada $x$ en el dominio $X$, existe un único $y$ en el recorrido $Y$.
    \item Dos pares $(x, y)$ y $(x, z)$ no pueden existir con el mismo $x$ y valores distintos de $y$.
\end{itemize}

\section*{Funciones. Definición formal como conjunto de pares ordenados}

Una función \( f \) es un conjunto de pares ordenados \( (x,y) \), donde:
\begin{itemize}
    \item Cada \( x \) en el dominio \( X \) tiene exactamente un \( y \) asociado.
    \item Dos pares \( (x,y) \) y \( (x,z) \) no pueden existir con el mismo \( x \) y valores distintos de \( y \).
    \item La imagen de  \( x \) se denota como \( f(x) \).
\end{itemize}

\subsubsection*{Dominio y recorrido}
\begin{itemize}
    \item El \textbf{dominio} de \( f \) es el conjunto de todos los \( x \) que aparecen como primeros elementos en los pares \( (x,y) \).
    \item El \textbf{recorrido} de \( f \) es el conjunto de todos los \( y \) que aparecen como segundos elementos en los pares \( (x,y) \).
\end{itemize}

\subsubsection*{Representación de funciones}
\begin{itemize}
    \item Una función puede imaginarse como una tabla con dos columnas: una para los valores de \( x \) (dominio) y otra para los valores de \( y \) (recorrido).
    \item Para todo \( x \) en el dominio de \( f \), existe exactamente un \( y \) tal que \( (x,y) \in f \).
\end{itemize}

\subsubsection*{Teorema sobre la igualdad de funciones}
\begin{itemize}
    \item Tienen el mismo dominio.
    \item \( f(x) = g(x) \) para todo \( x \) en el dominio.
\end{itemize}
