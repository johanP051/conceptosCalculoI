\section{Análisis Numérico o Computacional}

\begin{enumerate}
    \item Use el método de la bisección para encontrar una aproximación a la raíz de la ecuación \(f(x)=x^3-x-2\) en el intervalo \([1,2]\) con una tolerancia de \(10^{-3}\).
    \item Aproxime la derivada de \(f(x)=e^x\) en \(x=1\) usando diferencias finitas progresivas con \(h=0.1\).
\end{enumerate}

Para revisar el código de la solución de este problema, consulte la carpeta \textit{analisisComputacional} del repositorio donde he subido esta tarea: \url{https://github.com/johanP051/conceptosCalculoI/blob/main/analisisComputacional/biseccion.py}.

\subsection*{Programa para la bisección}

\begin{lstlisting}[style=jupyter]
    import numpy as np
    from decimal import Decimal, getcontext
    
    getcontext().prec = 50  #Se le da una precisión alta como un Double float de C++
    
    class Biseccion():
        def __init__(self):
            self.coeficientes = []
    
        def insertar_ecuacion(self):
            gradoPolinomio = int(input("Inserte el grado del polinomio: "))
            self.coeficientes = []
    
            for i in reversed(range(gradoPolinomio + 1)):
                coeficiente = Decimal(input(f"Inserte el coeficiente de x^{i}: "))
                self.coeficientes.append(coeficiente)
        
        def insertarIntervalo(self):
            print("Ahora debe insertar los intervalos \n")
            self.a = Decimal(input("Inserte el valor de [a]: "))
            self.b = Decimal(input("Inserte el valor de [b]: "))
    
        def evaluar_funcion(self, x):
            x = Decimal(x)
            return np.polyval(self.coeficientes, x)
        
        def calcularRaiz(self):
            self.x_inf = self.a
            self.x_sup = self.b
    
            while True:
                x_a = (self.x_sup + self.x_inf) / 2
                signo = self.evaluar_funcion(self.x_inf) * self.evaluar_funcion(x_a)
    
                ## Haciendo el cambio de límites de intervalos
                if signo < 0:
                    self.x_sup = x_a
                else:
                    self.x_inf = x_a
    
                error = self.evaluar_funcion(x_a)
    
                if abs(error) <= Decimal('1E-3'):
                    print(f"Resultado de la raíz: {x_a}")
                    break
                print(x_a)
    
    ec = Biseccion()
    ec.insertar_ecuacion()
    ec.insertarIntervalo()
    ec.calcularRaiz()
\end{lstlisting}

\subsection*{Programa para la derivada}
\begin{lstlisting}[style=jupyter]
    import numpy as np

    class DiferenciasFinitas():
        def __init__(self):
            pass
    
        def insertar_valores(self):
            self.x = float(input("Inserte el valor de x a evaluar en la derivada: "))
            self.h = float(input("Inserte el valor de h: "))
    
        def funcion(self, x):
            return np.exp(x)
    
        def derivada_Orden1(self, x, h):
            # f'(x) ≈ (f(x+h) - f(x)) / (h)
            return (self.funcion(x + h) - self.funcion(x)) / (h)
    
        def derivada_Orden4(self, x, h):
            # f'(x) ≈ [ -25f(x) + 48f(x + h) - 36f(x + 2h) + 16f(x + 3h) -3f(x + 4h) ] / (12h)
            return (-25*self.funcion(x) + 48*self.funcion(x + h) - 36*self.funcion(x + 2*h) + 16*self.funcion(x + 3*h) - 3*self.funcion(x + 4*h)) / (12*h) 
    
        def calcular_derivadas(self):
            d2 = self.derivada_Orden1(self.x, self.h)
            d4 = self.derivada_Orden4(self.x, self.h)
    
            #Para verificar se revisa que d/dx e^x sea aproximadamente e^x, como lo dice la teoria
            derivada_exacta = self.funcion(self.x)
            
            print(f"\nPara f(x) = e^x evaluada en 1 = {self.x}:")
            print(f"Derivada (diferencias finitas progresivas orden 2): {d2}")
            print(f"Derivada (diferencias finitas progresivas orden 4): {d4}")
            print(f"Derivada exacta: {derivada_exacta}")
    
    dx = DiferenciasFinitas()
    dx.insertar_valores()
    dx.calcular_derivadas()
\end{lstlisting}