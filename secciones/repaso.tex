\section{Repaso}

La intención de las siguientes preguntas es repasar y recordar algunos conceptos básicos que debe tener en cuenta para desarrollar esta tarea e iniciar el curso de cálculo integral.

\begin{itemize}
    \item \textit{Pregunta 1} ¿Qué es una función real, el dominio, el rango y la gráfica de una función real?
    \item \textit{Pregunta 2} ¿Cuál es la definición de valor absoluto?
    \item \textit{Pregunta 3} ¿Qué relación existe entre el concepto de límite de una sucesión o una función con el concepto de valor absoluto?
    \item \textit{Pregunta 4} ¿Qué significa que exista y calcular, el límite de una función real en un punto de su dominio?
    \item \textit{Pregunta 5} ¿Qué es la derivada de una función real en un punto de su dominio? Explique en sus palabras el significado de tasa de cambio instantáneo, qué es la recta tangente a la gráfica de una función en un punto de su dominio, si todas las funciones reales son diferenciables y si el concepto de derivada es global o local en el dominio.
    \item \textit{Pregunta 6} ¿Cuál es la relación entre el concepto de límite y el de derivada?
\end{itemize}