\section{Problemas Rutinarios}

Se espera que las siguientes actividades le permitan comprender, repasar y utilizar propiedades y métodos analíticos que aprendió en el curso de Cálculo Diferencial:

\begin{enumerate}
    \item Determine la ecuación de la recta tangente a la función \(f(x)=x^2+3x-5\) en \(x=1\).
    \item Encuentre los puntos críticos y clasifíquelos como máximos, mínimos o puntos de silla para la función \(f(x)=x^3-6x^2+9x+2\).
    \item Analice la continuidad de la función:
    $$
    f(x)=\begin{cases}
    x^2-1, & x<2, \\
    3x-5, & x\geq 2.
    \end{cases}
    $$
    \item Use herramientas de cálculo para hacer un \emph{sketch} de la gráfica de las siguientes funciones, además, describa el paso a paso para llegar a su \emph{sketch}:
    \begin{enumerate}
        \item \(f(x)=\frac{x}{x^2-1}\),
        \item \(f(x)=\frac{1}{(x-1)(x-3)}\),
        \item \(f(x)=x+\frac{1}{x^2}\).
    \end{enumerate}
\end{enumerate}